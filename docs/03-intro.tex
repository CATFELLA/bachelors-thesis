\section*{ВВЕДЕНИЕ}
\addcontentsline{toc}{section}{ВВЕДЕНИЕ}

В 2019 году процессоры архитектуры ARM составляли 34\% от рынка процессоров, при этом занимая 90\% рынка мобильных процессоров \cite{arm_report}. С появлением процессоров M1 компании Apple большое число людей стали пользоваться компьютерами на основе архитектуры ARM в качестве персональных компьютеров. (чето написать типа им нужно использовать проприетарный софт который нельзя самому для себя пересобрать)

Программы собранные под архитектуру x86 не работают на таких компьютерах, необходим или статический транслятор, такой как Rosetta 2, или виртуальная машина поддерживающая необходимую архитектуру. Rosetta 2 не транслирует программы не предназначенные для macOS, для запуска программ созданных для Windows или Linux нужно использовать виртуальную машину. Еще одно ограничение статической трансляции --- наличие самомодифицирующегося кода и динамических библиотек, таким образом использование только статической трансляции не запустит любую программу. \cite{fast_bin}

Системы трансляции они крутые, они со всем справляются но у них есть родовая травма, поскольку трансляция это всегда дополнительный какой-то шаг они добавляют оверхед, задачей данной работы является исследовать причины оверхеда и попытаться найти пути этого оверхеда устранения. Можно привести пару ссылок что есть такая проблема оверхеда.

Для достижения поставленной цели необходимо решить следующие задачи:

\begin{itemize}[leftmargin=1.6\parindent]
	\item[---] бла;
	\item[---] бла;
	\item[---] бла;
	\item[---] бла;
	\item[---] бла.
\end{itemize}

\pagebreak