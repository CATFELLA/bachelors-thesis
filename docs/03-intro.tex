\section*{ВВЕДЕНИЕ}
\addcontentsline{toc}{section}{ВВЕДЕНИЕ}

Процессоры архитектуры ARM занимают большую долю рынка, еще в 2015 году они составляли 35\% от рынка процессоров, однако в основном они использовались в портативных устройствах \cite{arm_report}.  С появлением процессоров M1 от компании Apple большое число людей начало пользоваться компьютерами на основе архитектуры ARM в домашней обстановке (типа personal computers). Однако, программы собранные под архитектуру x86 не смогут работать на таких компьютерах, им необходим или транслятор, такой как Rosetta 2, или виртуальная машина поддерживающая необходимую архитектуру.

Виртуальные машины как правило используют динамическую трансляцию, поэтому они намного более чувствительны к ее скорости.

\pagebreak