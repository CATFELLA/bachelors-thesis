\section*{ВВЕДЕНИЕ}
\addcontentsline{toc}{section}{ВВЕДЕНИЕ}

Процессоры архитектуры ARM занимают большую долю рынка, еще в 2015 году они составляли 35\% от рынка процессоров, однако в основном они использовались в портативных устройствах \cite{arm_report}.  С появлением процессоров M1 от компании Apple большое число людей начало пользоваться компьютерами на основе архитектуры ARM в домашней обстановке. Однако, программы собранные под архитектуру x86 не смогут работать на таких компьютерах, им необходим или транслятор, такой как Rosetta 2, или виртуальная машина поддерживающая необходимую архитектуру. Несмотря на хорошие результаты, Rosetta 2 не может транслировать программы не предназначенные для macOS, для запуска программ созданных для Windows или Linux в основном необходимо использовать виртуальную машину. Еще одним ограничением статической трансляции является наличие самомодифицирующегося кода и динамических библиотек, таким образом использование только статической трансляции никогда не сможет запустить любую программу. \cite{fast_bin}

Виртуальные машины в основном используют динамическую трансляцию, поэтому они намного более чувствительны к ее скорости.

Цель данной работы – провести обзор методов применяемых в трансляции машинного кода на x86 в машинный код архитектуры ARM.

\pagebreak