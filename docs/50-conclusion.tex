\section*{ЗАКЛЮЧЕНИЕ}
\addcontentsline{toc}{section}{ЗАКЛЮЧЕНИЕ}

В рамках этой работы:
\begin{itemize}[leftmargin=1.6\parindent]
	\item [---] описаны инструменты и технологии используемые для трансляции;
	\item [---] проанализированы издержки трансляции;
	\item [---] был реализован оптимизирующий доступ к памяти проход над промежуточным представлением;
	\item [---] проход был встроен в динамический транслятор FEX;
	\item [---] проведено исследование результатов работы оптимизирующего прохода;
	\item [---] проведено исследование корректность работы оптимизирующего прохода.
\end{itemize}

Таким образом цель работы --- нахождение и устранение некоторых издержек трансляции была достигнута.

В результате рост производительности бенчмарка nbench для различных ядер составил:
\begin{itemize}[leftmargin=1.6\parindent]
	\item[---] Cortex A53 --- производительность в среднем выше в 0.42 процента, для FOURIER --- прирост 1 процент, IDEA --- 2.7 процента, HUFFMAN --- меньше процента;
	\item[---] Cortex A72 --- производительность в среднем выше в 1.77 процента, для FOURIER --- прирост 7.5 процента, IDEA --- 12.2 процента, HUFFMAN --- 4.2 процента;
	\item[---] Mongoose M2 --- производительность в среднем выше в 1.29 процента, для FOURIER --- прирост 4.4 процента, IDEA --- 6.8 процента, HUFFMAN --- 5.3 процента;
	\item[---] M1 Firestorm --- производительность в среднем выше в 1.76 процента, для FOURIER --- прирост 4.5 процента, IDEA --- меньше процента, HUFFMAN --- 5.7 процента;
\end{itemize}

В качестве дальнейшего развития могут быть предложены следующие
направления:

\begin{itemize}[leftmargin=1.6\parindent]
	\item [---] проверка состояния прочих регистров, кроме RBP;
	\item [---] интеграция в другие динамические трансляторы.
\end{itemize}

\pagebreak