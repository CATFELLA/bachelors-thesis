\section*{ПРИЛОЖЕНИЕ A}
\addcontentsline{toc}{section}{ПРИЛОЖЕНИЕ A}

\begin{code}
	\captionof{listing}{Проход оптимизации доступа к памяти}
	\label{code:args}
	\begin{minted}
		[
		frame=single,
		framerule=0.5pt,
		framesep=20pt,
		fontsize=\scriptsize,
		tabsize=4,
		linenos,
		numbersep=5pt,
		xleftmargin=10pt,
		]
		{text}
#include <FEXCore/Core/X86Enums.h>
#include <FEXCore/IR/IR.h>
#include <FEXCore/IR/IREmitter.h>
#include <FEXCore/IR/IntrusiveIRList.h>

#include <array>
#include <iostream>

#include "Interface/Core/OpcodeDispatcher.h"
#include "Interface/IR/PassManager.h"

namespace FEXCore::IR {
	
	enum RBP_STATE {
		NOT_CHANGED = (0b000 << 0),
		STACK = (0b001 << 0),
		NOT_STACK = (0b010 << 0),
	};
	
	struct BlockInfo {
		int State = NOT_CHANGED;
		std::set<OrderedNode*> StackNodes;
		std::set<OrderedNode*> UnStackNodes;
		std::vector<OrderedNode*> Predecessors;
		bool Visited = false;
	};
	
	class StackAccessTSORemovalPass final : public Pass {
		public:
		bool Run(IREmitter* IREmit) override;
		
		private:
		void CalculateControlFlowInfo(IREmitter* IREmit);
		int CalculateComplexState(const OrderedNode* TargetNode,
		const IRListView& CurrentIR);
		std::unordered_map<NodeID, BlockInfo> OffsetToBlockMap;
	};
	
	static bool HitsRegister(const int Register, IREmitter* IREmit,
	IROp_Header* Op) {
		std::queue<IROp_Header*> values{};
		values.push(Op);
		
		while (!values.empty()) {
			IROp_Header* ValOp = values.front();
			values.pop();
			
			switch (ValOp->Op) {
				case IR::OP_ADD:
				case IR::OP_SUB:
				case IR::OP_OR:
				case IR::OP_AND: {
					values.push(IREmit->GetOpHeader(ValOp->Args[0]));
					values.push(IREmit->GetOpHeader(ValOp->Args[1]));
					
					break;
				}
				case IR::OP_LOADREGISTER: {
					auto LocalOp = ValOp->CW<IR::IROp_LoadRegister>();
					
					if (LocalOp->Offset ==
					offsetof(FEXCore::Core::CPUState, gregs[Register]) &&
					LocalOp->Class == GPRClass) {
						return true;
					}
					
					break;
				}
				default: {
					break;
				}
			}
		}
		
		/* never hit the RSP register, so it's not a stack value being
		* loaded. */
		return false;
	}
	
	void StackAccessTSORemovalPass::CalculateControlFlowInfo(IREmitter* IREmit) {
		IRListView CurrentIR = IREmit->ViewIR();
		
		for (auto [BlockNode, BlockHeader] : CurrentIR.GetBlocks()) {
			BlockInfo* CurrentBlock =
			&OffsetToBlockMap.try_emplace(CurrentIR.GetID(BlockNode)).first->second;
			for (auto [CodeNode, IROp] : CurrentIR.GetCode(BlockNode)) {
				switch (IROp->Op) {
					case IR::OP_CONDJUMP: {
						auto Op = IROp->CW<IR::IROp_CondJump>();
						
						{
							auto Block =
							&OffsetToBlockMap.try_emplace(Op->TrueBlock.ID()).first->second;
							Block->Predecessors.emplace_back(BlockNode);
						}
						
						{
							auto Block = &OffsetToBlockMap.try_emplace(Op->FalseBlock.ID())
							.first->second;
							Block->Predecessors.emplace_back(BlockNode);
						}
						
						break;
					}
					case IR::OP_JUMP: {
						auto Op = IROp->CW<IR::IROp_Jump>();
						
						{
							auto Block =
							OffsetToBlockMap.try_emplace(Op->Header.Args[0].ID()).first;
							Block->second.Predecessors.emplace_back(BlockNode);
						}
						
						break;
					}
					case IR::OP_STOREREGISTER: {
						IROp_StoreRegister* Op = IROp->CW<IR::IROp_StoreRegister>();
						if (Op->Offset == offsetof(FEXCore::Core::CPUState,
						gregs[FEXCore::X86State::REG_RBP]) &&
						Op->Class == GPRClass) {
							if (HitsRegister(FEXCore::X86State::REG_RSP, IREmit,
							IREmit->GetOpHeader(Op->Value))) {
								CurrentBlock->State = STACK;
								CurrentBlock->StackNodes.emplace(CodeNode);
							} else {
								CurrentBlock->State = NOT_STACK;
								CurrentBlock->UnStackNodes.emplace(CodeNode);
							}
						}
						break;
					}
					default:
					break;
				}
			}
		}
	}
	
	// returns the combined state of all the predecessors and the block itself
	int StackAccessTSORemovalPass::CalculateComplexState(
	const OrderedNode* TargetNode, const IRListView& CurrentIR) {
		auto TargetPair = OffsetToBlockMap.find(CurrentIR.GetID(TargetNode));
		if (TargetPair == OffsetToBlockMap.end()) {
			std::cout << "TargetPair not found in offsets!" << std::endl;
			return NOT_STACK;
		}
		
		BlockInfo* TargetBlock = &TargetPair->second;
		
		if (TargetBlock->Visited || TargetBlock->Predecessors.size() == 0 ||
		TargetBlock->State != NOT_CHANGED) {
			return TargetBlock->State;
		}
		
		TargetBlock->Visited = true;
		int ResultingState = STACK;
		for (auto i : TargetBlock->Predecessors) {
			if (i == TargetNode) {
				continue;
			}
			int PredecessorState = CalculateComplexState(i, CurrentIR);
			
			if (PredecessorState == NOT_CHANGED) {
				ResultingState = NOT_CHANGED;
			} else if (PredecessorState == NOT_STACK) {
				ResultingState = NOT_STACK;
				break;
			}
		}
		
		TargetBlock->State = ResultingState;
		
		return TargetBlock->State;
	}
	
	bool StackAccessTSORemovalPass::Run(IREmitter* IREmit) {
		CalculateControlFlowInfo(IREmit);
		
		IRListView CurrentIR = IREmit->ViewIR();
		bool Changed = false;
		
		for (auto [BlockNode, BlockHeader] : CurrentIR.GetBlocks()) {
			auto CurrentBlockPair = OffsetToBlockMap.find(CurrentIR.GetID(BlockNode));
			if (CurrentBlockPair == OffsetToBlockMap.end()) {
				std::cout << "Block not found in offsets" << std::endl;
				continue;
			}
			
			int StackStatus = STACK;
			BlockInfo* CurrentBlock = &CurrentBlockPair->second;
			for (auto CodeNode : CurrentBlock->Predecessors) {
				int PredecessorState = CalculateComplexState(CodeNode, CurrentIR);
				
				if (PredecessorState != STACK) {
					StackStatus = PredecessorState;
					break;
				}
			}
			
			for (auto [CodeNode, IROp] : CurrentIR.GetCode(BlockNode)) {
				if (CurrentBlock->StackNodes.contains(CodeNode)) {
					StackStatus = STACK;
				}
				if (CurrentBlock->UnStackNodes.contains(CodeNode)) {
					StackStatus = NOT_STACK;
				}
				if (IROp->Op == IR::OP_LOADMEMTSO && StackStatus == STACK) {
					if (HitsRegister(FEXCore::X86State::REG_RBP, IREmit,
					IREmit->GetOpHeader(IROp->Args[0])) ||
					HitsRegister(FEXCore::X86State::REG_RSP, IREmit,
					IREmit->GetOpHeader(IROp->Args[0]))) {
						IROp->Op = IR::OP_LOADMEM;
						Changed = true;
					}
				}
				if (IROp->Op == IR::OP_STOREMEMTSO && StackStatus == STACK) {
					if (HitsRegister(FEXCore::X86State::REG_RBP, IREmit,
					IREmit->GetOpHeader(IROp->Args[1])) ||
					HitsRegister(FEXCore::X86State::REG_RSP, IREmit,
					IREmit->GetOpHeader(IROp->Args[1]))) {
						IROp->Op = IR::OP_STOREMEM;
						Changed = true;
					}
				}
			}
			
			CurrentBlock->Visited = true;
			CurrentBlock->State = StackStatus;
		}
		
		return Changed;
	}
	
	std::unique_ptr<Pass> CreateStackAccessTSORemovalPass() {
		return std::make_unique<StackAccessTSORemovalPass>();
	}
}  // namespace FEXCore::IR
	\end{minted}
\end{code}

\begin{comment}
\begin{Verbatim}[fontsize=\footnotesize]
Hi. Thanks for your interest in Box! I have answered your question below,
to make things easier to read.
As a general design principle for Box, I try to get the Dynarec not too
complex, but still trying to have decent translated code. Also, I try to be
able to get back to an x86 instruction from an ARM generated code to be
able to handle Signal properly.

Sebastien.

Le mer. 8 d=C3=A9c. 2021 =C3=A0 13:34, Mikhail Nitenko <mnitenko@gmail.com>=
a =C3=A9crit :

> Hello!
>
> I=E2=80=99m a final-year student interested in dynamic binary translation=
,
> specifically in x86 to ARM. I am writing a bachelor thesis about it
> and wanted to explore the current optimizations related to
> translation and wanted to write to you since your project seems to be
> quite advanced.
>
> I read the =C2=ABInner workings=C2=BB and =C2=ABA deep dive into library =
wrapping=C2=BB
> articles and wondered about some things, specifically:
> * How do you determine where the translation block ends? You said
> =C2=ABthere are a lot of possibilities here=C2=BB, from reading QEMU docs=
I
> would imagine that one of them would be a CPU state change that is
> impossible to determine ahead of time.
>
A block is a contiguous array of x86 instructions. A block ends when the
last instruction has no "next" instruction (like a jump, call or ret) , and
when that instruction is not referenced by a jump from the current block.
There are a few exception: multi-bytes NOP are handled, and an unknown
instruction stop the block unconditionally

* How much memory is allocated for each x86 instruction during pass 0?
> Is there some kind of table that determines that?
>
The memory is dynamically allocated in pass 0. So it will use as much
memory as needed.

* JITs (mono) and self-modifying code. I know that emulating
> self-modifying code in x86 is difficult, QEMU for example detects
> writes to pages and invalidates translated code at that page and all
> pages that follow, are you doing the same? Is there some document that
> I can read to learn about problems related to JIT (mono, which I
> assume is open-source .NET)?
>
I'm doing something similar, yes. All pages from which x86 code has been
translated are write protected. Once a write occurs, all blocks generated
from this page are marked dirty, and the jump table from those blocks are
invalidated (to avoid automatic jump from block to block to this one). Once
x86 code wants to run a "dirty" block, a crc32 of the memory is runned and
compared to the crc computed when creating the block. depending if it's the
same or non, the block is marked as clean (and the page protected again) or
deleted and a new one is generated (and the page also protected again)

* Am I correct in understanding that dynarec is used to translate x86
> instructions into ARM instructions and x86 emulation is code in C that
> emulates an instruction?
>
Box86 and Box64 include an Interpreter, coded in C, that can interpret x86
code on any architecture, and a Dynarec, actually only available for ARM.
The Dynarec is coded in C also, with very little assembly file (just the
prolog / epilog of a function call basically). The Dynarec will use
"Emitter", (ab)using C macro, to generate actual arm opcodes.

>
>
> Also, maybe you would be able to point me to some other resource
> where I learn more about translation optimizations? Any help would be
> greatly appreciated!
>
You should have a look at FEX project. This project has a huge emphasis on
the dynarec and code translation optimisation
\end{Verbatim}

\pagebreak

\section*{ПРИЛОЖЕНИЕ Б}
\addcontentsline{toc}{section}{ПРИЛОЖЕНИЕ Б}

\begin{Verbatim}[fontsize=\footnotesize]
Oh, hey there.

For your specific questions. The pass manager is purely a construct
that manages some pass classes and ensures optimization passes are run
over the IR in correct order.
This can be found here:
https://github.com/FEX-Emu/FEX/blob/main/External/FEXCore/Source/Interface/IR/
PassManager.cpp
The same equivalent can be found in other compiler projects like LLVM.

Self-modifying code is currently semi-nonworking under FEX.
We have three modes of operation:
No SMC - Assumes zero code invalidation, highly dangerous.
mmap based invalidation - On mmap and munmap, ensure any overlapping
executable regions get code invalidated. Fixes issues where libraries
are loaded and need invalidation
per-instruction validation - This is the worst case situation, we emit
an instruction check at every.single. emulated instruction. Very
/very/ slow, only for debugging purposes.

We have plans to implement write protecting based invalidation but
haven't had the time to get around to it. It's pretty much the only
good way of detecting SMC.

The full list of passes can be seen in the PassManager link with
`AddDefaultPasses`, `AddDefaultValidationPasses`, and
`InsertRegisterAllocationPass`.
With implementations living at:
https://github.com/FEX-Emu/FEX/tree/main/External/FEXCore/Source/Interface/IR/
Passes

Unlike a compiler, we've got some heavy deadline constraints for how
long we can take optimizing the code, and luckily the code will have
run through a compiler that does most of that anyway.
Some of the more "hidden" optimizations come from around the IR
optimization. Our IR is designed to look like AArch64 to some extent,
which means we don't really need something like a high level IR then a
machine level IR to get close to "optimal".
Additionally some of the memory allocations and container functions
around the IR optimize around code size limits, forward only temporary
allocator for smart CPU cache behaviour, linked lists that the
elements are packed tightly in most cases so forward data fetching
that the CPU does will be optimal.
Probably some other various bits.

I should also say that I'm not fully happy about FEX's codegen yet
either. There's a /lot/ of room for improvement since we miss a bunch
of optimization opportunities.
One of the big things that can help is our AOT compilation. We have
some minor work towards both x86->IR AOT and I'm in the process of
writing IR->Code AOT.
With AOT we can throw heavier optimizations at the IR that we can't do
in JIT time. Which will improve long term performance of the
application.

Currently our code generates a /bunch/ of redundant register moves for
example. ARM CPUs that have really good renaming support will be
inordinately faster than expected.

Some of the things to learn more about translation optimizations
mostly looks like reading compiler optimization techniques and picking
optimizations that don't take quadratic time complexity like that.
So anything compiler theory can help out JITs. There are of course
domain specific optimizations, which are what would help JITs the
most.
For applications using x87, you can optimize that to not use a
softfloat emulation for example.
\end{Verbatim}
\end{comment}

\pagebreak