\section*{РЕФЕРАТ}

Расчетно-пояснительная записка \pageref{LastPage} с., \totalfigures\ рис., \totaltables\ табл., 20 ист., 3 прил.

Объектом исследования данной работы является динамческая трансляция. Использование трансляции по своей природе несет дополнительные издержки при выполнении программы, как в памяти, так и в скорости выполнения. Целью этой работы является исследование причин дополнительных издержек и попытка нахождения пути устранения этих издержек.

Для достижения поставленной цели необходимо решить следующие задачи:

\begin{itemize}[leftmargin=1.6\parindent]
	\item [---] описать инструменты и технологии используемые для трансляции;
	\item [---] проанализировать издержки трансляции;
	\item [---] реализовать оптимизирующий проход над промежуточным представлением;
	\item [---] встроить проход в динамический транслятор FEX;
	\item [---] провести исследование результатов работы оптимизирующего прохода;
	\item [---] провести исследование корректности работы оптимизирующего прохода.
\end{itemize}

Поставленная цель достигнута: в ходе дипломной работы был разработан метод оптимизации трансляции доступа к памяти. Разработанный метод повышает скорость работы транслированных приложений с доступом к памяти через регистр RBP.

КЛЮЧЕВЫЕ СЛОВА

\textit{x86, динамическая трансляция, ARM, промежуточное представление, модели памяти}

\pagebreak