\section{Исследовательская часть}

В рамках дипломной работы было проведено исследование изменения результатов бенчмарка nbench с разработанным проходом. Результаты исследования представлены в этом разделе.

\section{Описание используемых данных}

Для проведения исследования используется бенчмарк nbench скомпилированный при помощи gcc в два разных бинарных файла: с флагом -O0 и с флагом -O3.

Динамический транслятор FEX запускается на SOC Rockchip RK3399, Exynos 8895 и Apple M1. Система работающая на RK3399 работает под управлением Linux; Exynos --- Android, сам запуск производится через chroot в окружение Debian; Apple M1 --- виртуальная машина с Ubuntu.

Из-за того что SOC RK3399 и Exynos 8895 используют архитектуру big.LITTLE у них одинаковые <<малые>> ядра --- Cortex A53. Больших различий между этими ядрами нет, поэтому в тестах использовались результаты работы этого ядра на RK3399. Раздельно представлены результаты бенчмарков для их <<больших>> ядер --- Cortex A72 для RK3399 и Mongoose M2 для Exynos 8895.

\section{Результаты исследования}

вернуть нули в графики

\begin{figure}[hbtp]
	\centering
	\begin{tikzpicture}  
		\begin{axis}  
			[ 
			height = 9cm,
			width = 12cm,
			ybar,
			enlarge x limits=0.15,
			enlarge y limits=0.06,
			ymax=760,
			ylabel={Итераций в секунду},
			symbolic x coords={Apple M1 Firestorm, Mongoose M2, Cortex A72},  
			xtick=data,  
			]  
			\addplot coordinates {(Apple M1 Firestorm, 244.86) (Cortex A72, 42.822) (Mongoose M2, 42.892)};
			\addplot coordinates {(Apple M1 Firestorm, 749.14) (Cortex A72, 175.83) (Mongoose M2, 273.33)};
			\legend{stock, patch}  
		\end{axis}  
	\end{tikzpicture}\\
	\caption{Тест NUMERIC SORT, nbench O0}
	\label{fig:speed}
\end{figure}

\begin{figure}[hbtp]
	\centering
	\begin{tikzpicture}  
		\begin{axis}  
			[ 
			height = 9cm,
			width = 12cm,
			ybar,
			enlarge x limits=0.15,
			enlarge y limits=0.145,
			ymax=1700,
			ylabel={Итераций в секунду},
			symbolic x coords={Apple M1 Firestorm, Mongoose M2, Cortex A72},  
			xtick=data,  
			]  
			\addplot coordinates {(Apple M1 Firestorm, 1445.4) (Cortex A72, 211.07) (Mongoose M2, 233.51)};
			\addplot coordinates {(Apple M1 Firestorm, 1664.2) (Cortex A72, 491.36) (Mongoose M2, 776.6)};
			\legend{stock, patch}  
		\end{axis}  
	\end{tikzpicture}\\
	\caption{Тест NUMERIC SORT, nbench O3}
	\label{fig:speed}
\end{figure}

\begin{figure}[hbtp]
	\centering
	\begin{tikzpicture}  
		\begin{axis}  
			[ 
			height = 9cm,
			width = 12cm,
			ybar,
			enlarge x limits=0.15,
			enlarge y limits=0.035,
			ymax=360,
			ylabel={Итераций в секунду},
			symbolic x coords={Apple M1 Firestorm, Mongoose M2, Cortex A72},  
			xtick=data,  
			]  
			\addplot coordinates {(Apple M1 Firestorm, 88.737) (Cortex A72, 12.993) (Mongoose M2, 12.125)};
			\addplot coordinates {(Apple M1 Firestorm, 356.68) (Cortex A72, 71.129) (Mongoose M2, 116.83)};
			\legend{stock, patch}  
		\end{axis}  
	\end{tikzpicture}\\
	\caption{Тест STRING SORT, nbench O0}
	\label{fig:speed}
\end{figure}

\begin{figure}[hbtp]
	\centering
	\begin{tikzpicture}  
		\begin{axis}  
			[ 
			height = 9cm,
			width = 12cm,
			ybar,
			enlarge x limits=0.15,
			enlarge y limits=0.03,
			ymax=750,
			ylabel={Итераций в секунду},
			symbolic x coords={Apple M1 Firestorm, Mongoose M2, Cortex A72},  
			xtick=data,  
			]  
			\addplot coordinates {(Apple M1 Firestorm, 166.49) (Cortex A72, 25.239) (Mongoose M2, 21.839)};
			\addplot coordinates {(Apple M1 Firestorm, 739.69) (Cortex A72, 120.41) (Mongoose M2, 220.3)};
			\legend{stock, patch}  
		\end{axis}  
	\end{tikzpicture}\\
	\caption{Тест STRING SORT, nbench O3}
	\label{fig:speed}
\end{figure}

\begin{figure}[hbtp]
	\centering
	\begin{tikzpicture}  
		\begin{axis}  
			[ 
			height = 9cm,
			width = 12cm,
			ybar,
			enlarge x limits=0.15,
			enlarge y limits=0.028,
			ymax=336230000,
			ylabel={Итераций в секунду},
			symbolic x coords={Apple M1 Firestorm, Mongoose M2, Cortex A72},  
			xtick=data,  
			]  
			\addplot coordinates {(Apple M1 Firestorm, 70700000) (Cortex A72, 9081300) (Mongoose M2, 10669000)};
			\addplot coordinates {(Apple M1 Firestorm, 336290000) (Cortex A72, 50998000) (Mongoose M2, 80384000)};
			\legend{stock, patch}  
		\end{axis}  
	\end{tikzpicture}\\
	\caption{Тест BITFIELD, nbench O0}
	\label{fig:speed}
\end{figure}

\begin{figure}[hbtp]
	\centering
	\begin{tikzpicture}  
		\begin{axis}  
			[ 
			height = 9cm,
			width = 12cm,
			ybar,
			enlarge x limits=0.15,
			enlarge y limits=0.13,
			ymax=489540000,
			ylabel={Итераций в секунду},
			symbolic x coords={Apple M1 Firestorm, Mongoose M2, Cortex A72},  
			xtick=data,  
			]  
			\addplot coordinates {(Apple M1 Firestorm, 169440000) (Cortex A72, 56259000) (Mongoose M2, 87132000)};
			\addplot coordinates {(Apple M1 Firestorm, 489440000) (Cortex A72, 210660000) (Mongoose M2, 297220000)};
			\legend{stock, patch}  
		\end{axis}  
	\end{tikzpicture}\\
	\caption{Тест BITFIELD, nbench O3}
	\label{fig:speed}
\end{figure}

\begin{figure}[hbtp]
	\centering
	\begin{tikzpicture}  
		\begin{axis}  
			[ 
			height = 9cm,
			width = 12cm,
			ybar,
			enlarge x limits=0.15,
			enlarge y limits=0.04,
			ymax=72,
			ylabel={Итераций в секунду},
			symbolic x coords={Apple M1 Firestorm, Mongoose M2, Cortex A72},  
			xtick=data,  
			]  
			\addplot coordinates {(Apple M1 Firestorm, 17.174) (Mongoose M2, 3.6548) (Cortex A72, 2.7173)};
			\addplot coordinates {(Apple M1 Firestorm, 70.873) (Mongoose M2, 19.727) (Cortex A72, 12.634)};
			\legend{stock, patch}  
		\end{axis}  
	\end{tikzpicture}\\
	\caption{Тест FP EMULATION, nbench O0}
	\label{fig:speed}
\end{figure}

\begin{figure}[hbtp]
	\centering
	\begin{tikzpicture}  
		\begin{axis}  
			[ 
			height = 9cm,
			width = 12cm,
			ybar,
			enlarge x limits=0.15,
			enlarge y limits=0.068,
			ymax=650,
			ylabel={Итераций в секунду},
			symbolic x coords={Apple M1 Firestorm, Mongoose M2, Cortex A72},  
			xtick=data,  
			]  
			\addplot coordinates {(Apple M1 Firestorm, 461.94) (Mongoose M2, 66.567) (Cortex A72, 41.355)};
			\addplot coordinates {(Apple M1 Firestorm, 643.12) (Mongoose M2, 235.04) (Cortex A72, 140.37)};
			\legend{stock, patch}  
		\end{axis}  
	\end{tikzpicture}\\
	\caption{Тест FP EMULATION, nbench O3}
	\label{fig:speed}
\end{figure}

\begin{figure}[hbtp]
	\centering
	\begin{tikzpicture}  
		\begin{axis}  
			[ 
			height = 9cm,
			width = 12cm,
			ybar,
			enlarge x limits=0.15,
			enlarge y limits=0.08,
			ymax=73385,
			ylabel={Итераций в секунду},
			symbolic x coords={Apple M1 Firestorm, Mongoose M2, Cortex A72},  
			xtick=data,  
			]  
			\addplot coordinates {(Apple M1 Firestorm, 31195) (Mongoose M2, 5334.9) (Cortex A72, 7203)};
			\addplot coordinates {(Apple M1 Firestorm, 72385) (Mongoose M2, 19699) (Cortex A72, 14606)};
			\legend{stock, patch}  
		\end{axis}  
	\end{tikzpicture}\\
	\caption{Тест FOURIER, nbench O0}
	\label{fig:speed}
\end{figure}

\begin{figure}[hbtp]
	\centering
	\begin{tikzpicture}  
		\begin{axis}  
			[ 
			height = 9cm,
			width = 12cm,
			ybar,
			enlarge x limits=0.15,
			enlarge y limits=0.09,
			ymax=83508,
			ylabel={Итераций в секунду},
			symbolic x coords={Apple M1 Firestorm, Mongoose M2, Cortex A72},  
			xtick=data,  
			]  
			\addplot coordinates {(Apple M1 Firestorm, 43357) (Mongoose M2, 6834.1) (Cortex A72, 9892.5)};
			\addplot coordinates {(Apple M1 Firestorm, 83008) (Mongoose M2, 22571) (Cortex A72, 17026)};
			\legend{stock, patch}  
		\end{axis}  
	\end{tikzpicture}\\
	\caption{Тест FOURIER, nbench O3}
	\label{fig:speed}
\end{figure}

\begin{figure}[hbtp]
	\centering
	\begin{tikzpicture}  
		\begin{axis}  
			[ 
			height = 9cm,
			width = 12cm,
			ybar,
			enlarge x limits=0.15,
			enlarge y limits=0.065,
			ymax=19,
			ylabel={Итераций в секунду},
			symbolic x coords={Apple M1 Firestorm, Mongoose M2, Cortex A72},  
			xtick=data,  
			]  
			\addplot coordinates {(Apple M1 Firestorm, 6.0509) (Mongoose M2, 1.1416) (Cortex A72, 1.2583)};
			\addplot coordinates {(Apple M1 Firestorm, 18.089) (Mongoose M2, 7.2608) (Cortex A72, 4.1153)};
			\legend{stock, patch}  
		\end{axis}  
	\end{tikzpicture}\\
	\caption{Тест ASSIGNMENT, nbench O0}
	\label{fig:speed}
\end{figure}

\begin{figure}[hbtp]
	\centering
	\begin{tikzpicture}  
		\begin{axis}  
			[ 
			height = 9cm,
			width = 12cm,
			ybar,
			enlarge x limits=0.15,
			enlarge y limits=0.11,
			ymax=65,
			ylabel={Итераций в секунду},
			symbolic x coords={Apple M1 Firestorm, Mongoose M2, Cortex A72},  
			xtick=data,  
			]  
			\addplot coordinates {(Apple M1 Firestorm, 60.332) (Mongoose M2, 6.2922) (Cortex A72, 9.7657)};
			\addplot coordinates {(Apple M1 Firestorm, 63.124) (Mongoose M2, 17.205) (Cortex A72, 15.908)};
			\legend{stock, patch}  
		\end{axis}  
	\end{tikzpicture}\\
	\caption{Тест ASSIGNMENT, nbench O3}
	\label{fig:speed}
\end{figure}

\begin{figure}[hbtp]
	\centering
	\begin{tikzpicture}  
		\begin{axis}  
			[ 
			height = 9cm,
			width = 12cm,
			ybar,
			enlarge x limits=0.15,
			enlarge y limits=0.099,
			ymax=4580,
			ylabel={Итераций в секунду},
			symbolic x coords={Apple M1 Firestorm, Mongoose M2, Cortex A72},  
			xtick=data,  
			]  
			\addplot coordinates {(Apple M1 Firestorm, 2028.3) (Mongoose M2, 463.33) (Cortex A72, 401.68)};
			\addplot coordinates {(Apple M1 Firestorm, 4480.3) (Mongoose M2, 1076.2) (Cortex A72, 780)};
			\legend{stock, patch}  
		\end{axis}  
	\end{tikzpicture}\\
	\caption{Тест IDEA, nbench O0}
	\label{fig:speed}
\end{figure}

\begin{figure}[hbtp]
	\centering
	\begin{tikzpicture}  
		\begin{axis}  
			[ 
			height = 9cm,
			width = 12cm,
			ybar,
			enlarge x limits=0.15,
			enlarge y limits=0.30,
			ymax=7015,
			ylabel={Итераций в секунду},
			symbolic x coords={Apple M1 Firestorm, Mongoose M2, Cortex A72},  
			xtick=data,  
			]  
			\addplot coordinates {(Apple M1 Firestorm, 7036.8) (Mongoose M2, 1680) (Cortex A72, 1660.4)};
			\addplot coordinates {(Apple M1 Firestorm, 6999.7) (Mongoose M2, 3510.4) (Cortex A72, 1956.9)};
			\legend{stock, patch}  
		\end{axis}  
	\end{tikzpicture}\\
	\caption{Тест IDEA, nbench O3}
	\label{fig:speed}
\end{figure}

\begin{figure}[hbtp]
	\centering
	\begin{tikzpicture}  
		\begin{axis}  
			[ 
			height = 9cm,
			width = 12cm,
			ybar,
			enlarge x limits=0.15,
			enlarge y limits=0.05,
			ymax=1800,
			ylabel={Итераций в секунду},
			symbolic x coords={Apple M1 Firestorm, Mongoose M2, Cortex A72},  
			xtick=data,  
			]  
			\addplot coordinates {(Apple M1 Firestorm, 459.81) (Mongoose M2, 90.36) (Cortex A72, 81.802)};
			\addplot coordinates {(Apple M1 Firestorm, 1791.2) (Mongoose M2, 450) (Cortex A72, 307.7)};
			\legend{stock, patch}  
		\end{axis}  
	\end{tikzpicture}\\
	\caption{Тест HUFFMAN, nbench O0}
	\label{fig:speed}
\end{figure}

\begin{figure}[hbtp]
	\centering
	\begin{tikzpicture}  
		\begin{axis}  
			[ 
			height = 9cm,
			width = 12cm,
			ybar,
			enlarge x limits=0.15,
			enlarge y limits=0.104,
			ymax=5900,
			ylabel={Итераций в секунду},
			symbolic x coords={Apple M1 Firestorm, Mongoose M2, Cortex A72},  
			xtick=data,  
			]  
			\addplot coordinates {(Apple M1 Firestorm, 3644.6) (Mongoose M2, 547.46) (Cortex A72, 570.37)};
			\addplot coordinates {(Apple M1 Firestorm, 5890.6) (Mongoose M2, 2076.6) (Cortex A72, 1240)};
			\legend{stock, patch}  
		\end{axis}  
	\end{tikzpicture}\\
	\caption{Тест HUFFMAN, nbench O3}
	\label{fig:speed}
\end{figure}

\begin{figure}[hbtp]
	\centering
	\begin{tikzpicture}  
		\begin{axis}  
			[ 
			height = 9cm,
			width = 12cm,
			ybar,
			enlarge x limits=0.15,
			enlarge y limits=0.04,
			ymax=646,
			ylabel={Итераций в секунду},
			symbolic x coords={Apple M1 Firestorm, Mongoose M2, Cortex A72},  
			xtick=data,  
			]  
			\addplot coordinates {(Apple M1 Firestorm, 168.8) (Mongoose M2, 23.784) (Cortex A72, 33.265)};
			\addplot coordinates {(Apple M1 Firestorm, 636.6) (Mongoose M2, 229.72) (Cortex A72, 164.16)};
			\legend{stock, patch}  
		\end{axis}  
	\end{tikzpicture}\\
	\caption{Тест LU DECOMPOSITION, nbench O0}
	\label{fig:speed}
\end{figure}

\begin{figure}[hbtp]
	\centering
	\begin{tikzpicture}  
		\begin{axis}  
			[ 
			height = 9cm,
			width = 12cm,
			ybar,
			enlarge x limits=0.15,
			enlarge y limits=0.055,
			ymax=2947,
			ylabel={Итераций в секунду},
			symbolic x coords={Apple M1 Firestorm, Mongoose M2, Cortex A72},  
			xtick=data,  
			]  
			\addplot coordinates {(Apple M1 Firestorm, 1240.5) (Mongoose M2, 151.81) (Cortex A72, 281.26)};
			\addplot coordinates {(Apple M1 Firestorm, 2847.6) (Mongoose M2, 1024.3) (Cortex A72, 580.25)};
			\legend{stock, patch}  
		\end{axis}  
	\end{tikzpicture}\\
	\caption{Тест LU DECOMPOSITION, nbench O3}
	\label{fig:speed}
\end{figure}

\newpage

я только что понял что наверное все это вообще не нужно )))

вот получше вариантик

\newpage

\begin{figure}[hbtp]
	\centering
	\begin{tikzpicture}  
		\begin{axis}  
			[ 
			height = 8cm,
			width = 16cm,
			ybar,
			enlarge x limits=0.15,
			enlarge y limits=0.97,
			ymax=11,
			ylabel={Счет},
			symbolic x coords={Apple M1 Firestorm, Mongoose M2, Cortex A72, Cortex A53},  
			xtick=data,  
			]  
			\addplot coordinates {(Apple M1 Firestorm, 5) (Mongoose M2, 8.968) (Cortex A72, 10.205) (Cortex A53, 8.795)};
			\addplot coordinates {(Apple M1 Firestorm, 5) (Mongoose M2, 9.084) (Cortex A72, 10.386) (Cortex A53, 8.832)};
			\legend{stock, patch}  
		\end{axis}  
	\end{tikzpicture}\\
	\caption{nbench O0}
	\label{fig:speed}
\end{figure}

\begin{figure}[hbtp]
	\centering
	\begin{tikzpicture}  
		\begin{axis}  
			[ 
			height = 8cm,
			width = 16cm,
			ybar,
			enlarge x limits=0.15,
			enlarge y limits=1.64,
			ymax=7700,
			ylabel={Счет},
			symbolic x coords={Apple M1 Firestorm, Mongoose M2, Cortex A72, Cortex A53},  
			xtick=data,  
			]  
			\addplot coordinates {(Apple M1 Firestorm, 12) (Mongoose M2, 5326.9) (Cortex A72, 7125.4) (Cortex A53, 3950)};
			\addplot coordinates {(Apple M1 Firestorm, 12) (Mongoose M2, 5559.8) (Cortex A72, 7661.9) (Cortex A53, 3989.2)};
			\legend{stock, patch}  
		\end{axis}  
	\end{tikzpicture}\\
	\caption{nbench FOURIER O0}
	\label{fig:speed}
\end{figure}

\begin{figure}[hbtp]
	\centering
	\begin{tikzpicture}  
		\begin{axis}  
			[ 
			height = 8cm,
			width = 16cm,
			ybar,
			enlarge x limits=0.15,
			enlarge y limits=1.64,
			ymax=460,
			ylabel={Счет},
			symbolic x coords={Apple M1 Firestorm, Mongoose M2, Cortex A72, Cortex A53},  
			xtick=data,  
			]  
			\addplot coordinates {(Apple M1 Firestorm, 12) (Mongoose M2, 462.76) (Cortex A72, 402.74) (Cortex A53, 272.31)};
			\addplot coordinates {(Apple M1 Firestorm, 12) (Mongoose M2, 494.46) (Cortex A72, 451.94) (Cortex A53, 279.77)};
			\legend{stock, patch}  
		\end{axis}  
	\end{tikzpicture}\\
	\caption{nbench IDEA O0}
	\label{fig:speed}
\end{figure}

\begin{figure}[hbtp]
	\centering
	\begin{tikzpicture}  
		\begin{axis}  
			[ 
			height = 8cm,
			width = 16cm,
			ybar,
			enlarge x limits=0.15,
			enlarge y limits=1.64,
			ymax=87,
			ylabel={Счет},
			symbolic x coords={Apple M1 Firestorm, Mongoose M2, Cortex A72, Cortex A53},  
			xtick=data,  
			]  
			\addplot coordinates {(Apple M1 Firestorm, 12) (Mongoose M2, 90.183) (Cortex A72, 83.382) (Cortex A53, 85.976)};
			\addplot coordinates {(Apple M1 Firestorm, 12) (Mongoose M2, 95.002) (Cortex A72, 86.868) (Cortex A53, 86.536)};
			\legend{stock, patch}  
		\end{axis}  
	\end{tikzpicture}\\
	\caption{nbench HUFFMAN O0}
	\label{fig:speed}
\end{figure}

\newpage

Таким образом по результатам замеров видно что проход работает только на неоптимизированном бенчмарке (собранном с флагом -O0), также стоит заметить что для Apple M1 рост в производительности не так велик.

На O3 рост производительности незначителен, это связано с тем что регистр RBP, при оптимизации, используется в качестве регистра общего назначения.

Рост в производительности составил:
\begin{itemize}[leftmargin=1.6\parindent]
	\item[---] Cortex A53 --- ;
	\item[---] Cortex A72 --- ;
	\item[---] ещо другие ядра;
\end{itemize}

Более высокий рост производительности для старых ядер связан с затратностью барьерных операций с памятью, новые поколения ядер уменьшают затратность таких операций.

\section{Тестирование на надежность}

Главной причиной разработки алгоритма по оптимизации доступа к памяти являлась слабая модель архитектуры ARM, что при трансляции многопоточных приложений приводило к ошибкам.

Для проверки корректной работы прохода использовался tst-cond16.c --- тест библиотеки libc. Этот тест создает 8 потоков которые проводят операции с мьютексами. Стандартная версия транслятора и версия с оптимизирующим проходом этот тест проходит, без барьерных инструкций и с безусловной заменой барьерных операций связанных с регистром RBP тест не проходит.

\section{Вывод}

В результате исследования было установлено, что норм проход)

Также было частично доказано что барьерные инструкции доступа к памяти сильнее влияют на ранее выпущенные ядра ARM.