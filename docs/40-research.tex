\section{Исследовательская часть}

В рамках дипломной работы было проведено исследование изменения результатов бенчмарка nbench с разработанным проходом. Результаты исследования представлены в этом разделе.

\subsection{Описание используемых данных}

Для проведения исследования используется бенчмарк nbench скомпилированный при помощи gcc в два разных бинарных файла: с флагом -O0 и с флагом -O3.

В бенчмарке nbench реализовано 9 тестов:

\begin{itemize}[leftmargin=1.6\parindent]
	\item[---] Numeric sort --- сортировка массива 32-х битных чисел. Оценивается производительность работы с целыми числами.
	\item[---] String sort --- сортировка массива символов. Проверяется скорость работы с памятью.
	\item[---] Bitfield --- выполняет различные битовые операции. Например: очистить или выставить n бит, инвертировать число. Оценивается скорость выполнения таких операций.
	\item[---] Emulated floating-point --- небольшой эмулятор для работы с числами с плавающей запятой. Хорошо оценивает общую производительность.
	\item[---] Fourier coefficients --- рассчитывает коэффициенты Фурье. Оценивает производительность FPU, при этом память использует не активно.
	\item[---] Assignment algorithm --- решение задачи о назначениях. Работает с массивом, на результат влияет скорость последовательного доступа к памяти.
	\item[---] Huffman compression --- алгоритм Хаффмана. Операции с байтами, битами и с целыми числами. Считается хорошей оценкой общей производительности.
	\item[---] IDEA encryption --- алгоритм шифрования. Оценивает скорость исполнения кода.
	\item[---] LU Decomposition ---  алгоритм решения линейных уравнений. Работает с числами с плавающей запятой, использует только фундаментальные операции (+, -, *, /).
\end{itemize}

В итоге получаются INTEGER INDEX и FLOATING-POINT INDEX --- комбинированный результат бенчмарков.

Каждый бенчмарк запускается 5 раз, затем рассчитывается среднеквадратическое отклонение и доверительный интервал с уровнем доверия 0.95. Если длина полуинтервала меньше 1\% от абсолютного значения среднего, то замер времени считается успешным, иначе проводятся дополнительные замеры.

Динамический транслятор FEX запускается на SOC Rockchip RK3399, Exynos 8895 и Apple M1. Система работающая на RK3399 работает под управлением Linux; Exynos --- Android, сам запуск производится через chroot в окружение Debian; Apple M1 --- виртуальная машина с Ubuntu.

Из-за того что SOC RK3399 и Exynos 8895 используют архитектуру \\ big.LITTLE у них одинаковые <<малые>> ядра --- Cortex A53. Больших различий между этими ядрами нет, поэтому в тестах использовались результаты работы этого ядра на RK3399. Раздельно представлены результаты бенчмарков для их <<больших>> ядер --- Cortex A72 для RK3399 и Mongoose M2 для Exynos 8895.

Замеры времени выполнения проводились на версии FEX-2204.

\subsection{Результаты исследования}

На рисунках \ref{fig:o0}-\ref{fig:o3} представлены результаты замеров скорости выполнения nbench с использованием алгоритма и без.

\begin{figure}[hbtp]
	\centering
	\begin{tikzpicture}  
		\begin{axis}  
			[ 
			height = 8cm,
			width = 16cm,
			ybar,
			enlarge x limits=0.15,
			enlarge y limits=0.305,
			ymax=38,
			ylabel={Счет},
			symbolic x coords={M1 Firestorm, Mongoose M2, Cortex A72, Cortex A53},  
			xtick=data,  
			]  
			\addplot coordinates {(M1 Firestorm, 36.882) (Mongoose M2, 8.968) (Cortex A72, 10.205) (Cortex A53, 8.795)};
			\addplot coordinates {(M1 Firestorm, 37.532) (Mongoose M2, 9.084) (Cortex A72, 10.386) (Cortex A53, 8.832)};
			\legend{stock, patch}  
		\end{axis}  
	\end{tikzpicture}\\
	\caption{nbench O0}
	\label{fig:o0}
\end{figure}

\begin{figure}[hbtp]
	\centering
	\begin{tikzpicture}  
		\begin{axis}  
			[ 
			height = 8cm,
			width = 16cm,
			ybar,
			enlarge x limits=0.15,
			enlarge y limits=0.1375,
			ymax=32701,
			ylabel={Счет},
			symbolic x coords={M1 Firestorm, Mongoose M2, Cortex A72, Cortex A53},  
			xtick=data,  
			]  
			\addplot coordinates {(M1 Firestorm, 31202) (Mongoose M2, 5326.9) (Cortex A72, 7125.4) (Cortex A53, 3950)};
			\addplot coordinates {(M1 Firestorm, 32601) (Mongoose M2, 5559.8) (Cortex A72, 7661.9) (Cortex A53, 3989.2)};
			\legend{stock, patch}  
		\end{axis}  
	\end{tikzpicture}\\
	\caption{nbench FOURIER O0}
	\label{fig:speed}
\end{figure}

\begin{figure}[hbtp]
	\centering
	\begin{tikzpicture}  
		\begin{axis}  
			[ 
			height = 8cm,
			width = 16cm,
			ybar,
			enlarge x limits=0.15,
			enlarge y limits=0.155,
			ymax=2060.9,
			ylabel={Счет},
			symbolic x coords={M1 Firestorm, Mongoose M2, Cortex A72, Cortex A53},  
			xtick=data,  
			]  
			\addplot coordinates {(M1 Firestorm, 2035.4) (Mongoose M2, 462.76) (Cortex A72, 402.74) (Cortex A53, 272.31)};
			\addplot coordinates {(M1 Firestorm, 2052.9) (Mongoose M2, 494.46) (Cortex A72, 451.94) (Cortex A53, 279.77)};
			\legend{stock, patch}  
		\end{axis}  
	\end{tikzpicture}\\
	\caption{nbench IDEA O0}
	\label{fig:speed}
\end{figure}

\begin{figure}[hbtp]
	\centering
	\begin{tikzpicture}  
		\begin{axis}  
			[ 
			height = 8cm,
			width = 16cm,
			ybar,
			enlarge x limits=0.15,
			enlarge y limits=0.202,
			ymax=497.55,
			ylabel={Счет},
			symbolic x coords={M1 Firestorm, Mongoose M2, Cortex A72, Cortex A53},  
			xtick=data,  
			]  
			\addplot coordinates {(M1 Firestorm, 461.28) (Mongoose M2, 90.183) (Cortex A72, 83.382) (Cortex A53, 85.976)};
			\addplot coordinates {(M1 Firestorm, 487.55) (Mongoose M2, 95.002) (Cortex A72, 86.868) (Cortex A53, 86.536)};
			\legend{stock, patch}  
		\end{axis}  
	\end{tikzpicture}\\
	\caption{nbench HUFFMAN O0}
	\label{fig:speed}
\end{figure}

\begin{figure}[hbtp]
	\centering
	\begin{tikzpicture}  
		\begin{axis}  
			[ 
			height = 8cm,
			width = 16cm,
			ybar,
			enlarge x limits=0.15,
			enlarge y limits=0.207,
			ymax=138,
			ylabel={Счет},
			symbolic x coords={M1 Firestorm, Mongoose M2, Cortex A72, Cortex A53},  
			xtick=data,  
			]  
			\addplot coordinates {(M1 Firestorm, 136.162) (Mongoose M2, 28.307) (Cortex A72, 32.258) (Cortex A53, 23.67)};
			\addplot coordinates {(M1 Firestorm, 136.218) (Mongoose M2, 28.36) (Cortex A72, 32.318) (Cortex A53, 23.685)};
			\legend{stock, patch}  
		\end{axis}  
	\end{tikzpicture}\\
	\caption{nbench O3}
	\label{fig:o3}
\end{figure}

сделать baseline

штриховку сделать

\newpage

Таким образом по результатам замеров видно что проход работает только на не оптимизированном бенчмарке (собранном с флагом -O0), также стоит заметить что для Apple M1 рост в производительности на выделенных бенчмарках меньше.

На O3 рост производительности незначителен, это связано с тем что регистр RBP, при оптимизации, используется в качестве регистра общего назначения.

Рост в производительности составил:
\begin{itemize}[leftmargin=1.6\parindent]
	\item[---] Cortex A53 --- производительность в среднем выше в 0.42 процента, для FOURIER --- прирост 1 процент, IDEA --- 2.7 процента, HUFFMAN --- меньше процента;
	\item[---] Cortex A72 --- производительность в среднем выше в 1.77 процента, для FOURIER --- прирост 7.5 процента, IDEA --- 12.2 процента, HUFFMAN --- 4.2 процента;
	\item[---] Mongoose M2 --- производительность в среднем выше в 1.29 процента, для FOURIER --- прирост 4.4 процента, IDEA --- 6.8 процента, HUFFMAN --- 5.3 процента;
	\item[---] M1 Firestorm --- производительность в среднем выше в 1.76 процента, для FOURIER --- прирост 4.5 процента, IDEA --- меньше процента, HUFFMAN --- 5.7 процента;
\end{itemize}

Более высокий рост производительности для старых ядер связан с затратностью барьерных операций с памятью, новые поколения ядер уменьшают затратность таких операций. У Cortex A53, как было выяснено в аналитическом разделе, разница в скорости выполнения программ с барьерным доступом к памяти и без него мала, поэтому для этого ядра результаты меньше.

\subsection{Тестирование на корректность}

Главной причиной разработки алгоритма по оптимизации доступа к памяти являлась слабая модель архитектуры ARM, что при трансляции многопоточных приложений приводило к ошибкам.

Для проверки корректной работы прохода использовался tst-cond16.c --- тест библиотеки libc. Этот тест создает 8 потоков которые проводят операции с мьютексами. Стандартная версия транслятора и версия с оптимизирующим проходом этот тест проходит, без барьерных инструкций и с безусловной заменой барьерных операций связанных с регистром RBP тест не проходит.

\subsection{Вывод}

В результате исследования было установлено, что проход увеличивает производительность отдельных бенчмарков и в целом повышает скорость выполнения транслированного кода, но только если исходный код был собран без оптимизаций.

Также было частично доказано что барьерные инструкции доступа к памяти сильнее влияют на ранее выпущенные ядра ARM.
