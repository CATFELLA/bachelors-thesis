\section{Исследовательская часть}

В рамках дипломной работы было проведено исследование изменения результатов бенчмарка nbench с разработанным проходом. Результаты исследования представлены в этом разделе.

\section{Описание используемых данных}

Для проведения исследования используется бенчмарк nbench скомпилированный при помощи gcc в два разных бинарных файла: с флагом -O0 и с флагом -O3.

Динамический транслятор FEX запускается на SOC Rockchip RK3399, Exynos 8895 и Apple M1. Система работающая на RK3399 работает под управлением Linux; Exynos --- Android, сам запуск производится через chroot в окружение Debian; Apple M1 --- виртуальная машина с Ubuntu.

Из-за того что SOC RK3399 и Exynos 8895 используют архитектуру big.LITTLE у них одинаковые <<малые>> ядра --- Cortex A53. Больших различий между этими ядрами нет, поэтому в тестах использовались результаты работы этого ядра на RK3399. Раздельно представлены результаты бенчмарков для их <<больших>> ядер --- Cortex A72 для RK3399 и Mongoose M2 для Exynos 8895.

\section{Результаты исследования}

вернуть нули в графики

\begin{figure}[hbtp]
	\centering
	\begin{tikzpicture}  
		\begin{axis}  
			[ 
			height = 8cm,
			width = 16cm,
			ybar,
			enlarge x limits=0.15,
			enlarge y limits=0.97,
			ymax=11,
			ylabel={Счет},
			symbolic x coords={Apple M1 Firestorm, Mongoose M2, Cortex A72, Cortex A53},  
			xtick=data,  
			]  
			\addplot coordinates {(Apple M1 Firestorm, 5) (Mongoose M2, 8.968) (Cortex A72, 10.205) (Cortex A53, 8.795)};
			\addplot coordinates {(Apple M1 Firestorm, 5) (Mongoose M2, 9.084) (Cortex A72, 10.386) (Cortex A53, 8.832)};
			\legend{stock, patch}  
		\end{axis}  
	\end{tikzpicture}\\
	\caption{nbench O0}
	\label{fig:speed}
\end{figure}

\begin{figure}[hbtp]
	\centering
	\begin{tikzpicture}  
		\begin{axis}  
			[ 
			height = 8cm,
			width = 16cm,
			ybar,
			enlarge x limits=0.15,
			enlarge y limits=1.64,
			ymax=7700,
			ylabel={Счет},
			symbolic x coords={Apple M1 Firestorm, Mongoose M2, Cortex A72, Cortex A53},  
			xtick=data,  
			]  
			\addplot coordinates {(Apple M1 Firestorm, 12) (Mongoose M2, 5326.9) (Cortex A72, 7125.4) (Cortex A53, 3950)};
			\addplot coordinates {(Apple M1 Firestorm, 12) (Mongoose M2, 5559.8) (Cortex A72, 7661.9) (Cortex A53, 3989.2)};
			\legend{stock, patch}  
		\end{axis}  
	\end{tikzpicture}\\
	\caption{nbench FOURIER O0}
	\label{fig:speed}
\end{figure}

\begin{figure}[hbtp]
	\centering
	\begin{tikzpicture}  
		\begin{axis}  
			[ 
			height = 8cm,
			width = 16cm,
			ybar,
			enlarge x limits=0.15,
			enlarge y limits=1.64,
			ymax=460,
			ylabel={Счет},
			symbolic x coords={Apple M1 Firestorm, Mongoose M2, Cortex A72, Cortex A53},  
			xtick=data,  
			]  
			\addplot coordinates {(Apple M1 Firestorm, 12) (Mongoose M2, 462.76) (Cortex A72, 402.74) (Cortex A53, 272.31)};
			\addplot coordinates {(Apple M1 Firestorm, 12) (Mongoose M2, 494.46) (Cortex A72, 451.94) (Cortex A53, 279.77)};
			\legend{stock, patch}  
		\end{axis}  
	\end{tikzpicture}\\
	\caption{nbench IDEA O0}
	\label{fig:speed}
\end{figure}

\begin{figure}[hbtp]
	\centering
	\begin{tikzpicture}  
		\begin{axis}  
			[ 
			height = 8cm,
			width = 16cm,
			ybar,
			enlarge x limits=0.15,
			enlarge y limits=1.64,
			ymax=87,
			ylabel={Счет},
			symbolic x coords={Apple M1 Firestorm, Mongoose M2, Cortex A72, Cortex A53},  
			xtick=data,  
			]  
			\addplot coordinates {(Apple M1 Firestorm, 12) (Mongoose M2, 90.183) (Cortex A72, 83.382) (Cortex A53, 85.976)};
			\addplot coordinates {(Apple M1 Firestorm, 12) (Mongoose M2, 95.002) (Cortex A72, 86.868) (Cortex A53, 86.536)};
			\legend{stock, patch}  
		\end{axis}  
	\end{tikzpicture}\\
	\caption{nbench HUFFMAN O0}
	\label{fig:speed}
\end{figure}

add O3

\newpage

Таким образом по результатам замеров видно что проход работает только на неоптимизированном бенчмарке (собранном с флагом -O0), также стоит заметить что для Apple M1 рост в производительности не так велик.

На O3 рост производительности незначителен, это связано с тем что регистр RBP, при оптимизации, используется в качестве регистра общего назначения.

Рост в производительности составил:
\begin{itemize}[leftmargin=1.6\parindent]
	\item[---] Cortex A53 --- производительность в среднем выше в ;
	\item[---] Cortex A72 --- производительность целочисленных вычислений выше в 2.12 раз и в 1.57 раз для вычислений с плавающей точкой;
	\item[---] ещо другие ядра;
\end{itemize}

Более высокий рост производительности для старых ядер связан с затратностью барьерных операций с памятью, новые поколения ядер уменьшают затратность таких операций.

\section{Тестирование на надежность}

Главной причиной разработки алгоритма по оптимизации доступа к памяти являлась слабая модель архитектуры ARM, что при трансляции многопоточных приложений приводило к ошибкам.

Для проверки корректной работы прохода использовался tst-cond16.c --- тест библиотеки libc. Этот тест создает 8 потоков которые проводят операции с мьютексами. Стандартная версия транслятора и версия с оптимизирующим проходом этот тест проходит, без барьерных инструкций и с безусловной заменой барьерных операций связанных с регистром RBP тест не проходит.

\section{Вывод}

В результате исследования было установлено, что норм проход)

Также было частично доказано что барьерные инструкции доступа к памяти сильнее влияют на ранее выпущенные ядра ARM.
