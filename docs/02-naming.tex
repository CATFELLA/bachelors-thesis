\section*{ОПРЕДЕЛЕНИЯ, ОБОЗНАЧЕНИЯ И СОКРАЩЕНИЯ}

Транслятор --- программа или техническое средство, выполняющие трансляцию программы. \cite{translate}

Трансляция программы --- преобразование программы, представленной на одном языке программирования, в программу на другом языке равносильную первой. \cite{translate}

Статическая трансляция (Ahead-of-time (AOT)) --- трансляция проводящаяся до начала выполнения программы. Позволяет использовать более трудозатратные методы оптимизации.

Динамическая трансляция (Just-in-time (JIT)) --- трансляция выполняемая непосредственно во время выполнения программы. Имеет строгие ограничения на скорость трансляции.

Интерпретация --- построчный анализ, обработка и выполнение кода программы, в отличие от компиляции, где весь текст программы, перед запуском анализируется и транслируется в машинный или байт-код без её выполнения. \cite{interpret}

Блок трансляции --- непрерывная последовательность инструкций программы выделенная для трансляции.

Промежуточное представление ---  это специальный код используемый внутри компилятора или виртуальной машины для представления исходного кода. Предназначен для дальнейшей обработки, такой как оптимизация и трансляция в машинный код.

Статическое единственное присваивание (СЕП) --- промежуточное представление в котором каждой переменной значение присваивается один раз, переменные исходной программы разбиваются на версии, обычно с помощью добавления суффикса, таким образом, что каждое присваивание осуществляется уникальной версии переменной.

Эмуляция системы --- это совокупность логических и технических средств и ресурсов, направленных на полную имитацию технического устройства выбранной пользователем системы для максимально точного воспроизведения всех процессов, происходящих внутри эмулируемой системы. 

Эмуляция режима пользователя --- режим эмуляции поддерживающий запуск пользовательских приложений, не эмулирует полноценную систему, а использует ядро запускающей системы для выполнения необходимых системных задач.

RISC --- это вычислительная машина с упрощенной системой команд, которая обеспечивает увеличение скорости декодирования команд. \cite{it_dict}

Пролог --- машинный код в самом начале функции (процедуры, подпрограммы), выполняющий предваряющие действия по подготовке стека потока и регистров с целью их дальнейшего использования в теле функции.

Эпилог --- машинный код в самом конце функции (процедуры, подпрограммы), восстанавливающий резервирование пространства стека до начального состояния; восстанавливающий регистры до состояния, предшествовавшего вызову функции и производящий возврат управления из функции.

\pagebreak