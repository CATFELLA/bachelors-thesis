\section{Технологическая часть}

В данном разделе описываются средства разработки программного обеспечения, требования к вычислительной системе. Приводится структура разработанного приложения.

\subsection{Выбор средств разработки}

Из рассмотренных в аналитическом разделе трансляторов FEX лучше прочих подходит для реализации алгоритма, так как в нем присутствует СЕП.

\subsubsection{Выбор языка программирования}

Динамический транслятор FEX написан на языке C++. Для простого встраивания в проект было решено использовать язык C++ для реализации алгоритма.

\subsubsection{Сборка программного обеспечения?}

Для сборки проекта используется система сборки
CMake. Для компиляции реализации алгоритма необходимо добавить имя файла в соответствующий CMakeLists.txt.

\subsection{Требования к вычислительной системе}

Для запуска программного обеспечения требуется исходный код динамического транслятора FEX. Так как алгоритм реализует оптимизацию доступа к памяти с сохранением корректного выполнения многопоточных приложений требуется процессор с поддерживаемой архитектурой со слабым доступом к памяти и несколькими ядрами.

\subsection{Структура программного обеспечения}

Разработанное ПО реализовано как оптимизирующий проход над промежуточным представлением.

При инициализации транслятора вызывается функция AddDefaultPasses, внутри которой вызываются функции InsertPass регистрирующие оптимизирующие проходы в определенном порядке. Функция InsertPass принимает на вход std::unique\_ptr<Pass> --- указатель на экземпляр класса прохода. У класса обязательно должен быть метод Run принимающий на вход IREmitter и возвращающий bool --- был ли изменен код во время прохода.

\subsection{more?}

а написать про дампы + скрипт для дампов, объясняю как проблемы решал?

\subsection{Вывод}

В данном разделе были описаны средства разработки программного
обеспечения, требования к вычислительной системе. Была дана структура
разработанного приложения.

\pagebreak