\section{Технологическая часть}

В данном разделе описываются средства разработки программного обеспечения, требования к вычислительной системе. Приводится структура разработанного приложения.

\subsection{Выбор средств разработки}

Из рассмотренных в аналитическом разделе трансляторов FEX лучше прочих подходит для реализации алгоритма, так как в нем присутствует СЕП.

РАЗРАБАТЫВАЛОСЬ НА АРМЕ (НА ЧЕМ СОБИРАЛОСЬ И НА ЧЕМ РАЗРАБАТЫВАЛОСЬ)

\subsubsection{Выбор языка программирования}

Динамический транслятор FEX написан на языке C++. Для простого встраивания в проект было решено использовать язык C++ для реализации алгоритма.

ядро fex это монолитная библа, чтобы ее модифицировать надо писать код на цпп

\subsubsection{Сборка программного обеспечения}

Для сборки проекта используется система сборки
CMake. Для компиляции реализации алгоритма необходимо добавить имя файла в соответствующий CMakeLists.txt.

дерево зависимостей fex (cmake -> make -> graphwiz)

че за симейк какие цели используются

собирается нинжей, а почему (желательно написать)

какой командой сбилдить

распростроняется дипчик в виде патча

патч применяется с помощью, например, git am

написать на какой верссии я писал (FEX-2204, какой тег написать)

фул скрипт сборки с патчем

\subsection{Требования к вычислительной системе}

Для запуска программного обеспечения требуется исходный код динамического транслятора FEX. Так как алгоритм реализует оптимизацию доступа к памяти с сохранением корректного выполнения многопоточных приложений для прироста в скорости требуется процессор с поддерживаемой архитектурой со слабым доступом к памяти и несколькими ядрами. (НЕ ТОЛЬКО АРМ НО И НАПРИМЕР РИСК 5)

полученный фех ехе запускать на арме, но можно разрабатывать его и кросс компайлом (типа среда выполнения требует арм, а среда разработки нет). можно попробовать расписать как скросс компилировать

можно попробовать mtune

\subsection{Структура программного обеспечения}

Разработанное ПО реализовано как оптимизирующий проход над промежуточным представлением.

При инициализации транслятора вызывается функция AddDefaultPasses, внутри которой вызываются функции InsertPass регистрирующие оптимизирующие проходы в определенном порядке. Функция InsertPass принимает на вход std::unique\_ptr<Pass> --- указатель на экземпляр класса прохода. У класса обязательно должен быть метод Run принимающий на вход IREmitter и возвращающий bool --- был ли изменен код во время прохода.

вывести tree над папкой

запихнуть патч

написать про tst-cond16

ls патч, инсталлер и ридми

\subsection{more?}

а написать про дампы + скрипт для дампов, объясняю как проблемы решал?

так я узнал вот что превращаются TSO инструкции

проверить генерятся ли доки (ключ)

ключ отключения всех оптимизаций

\subsection{как контрибьютить}

кто разработчики, как контрибьютить и все такое

\subsection{Вывод}

В данном разделе были описаны средства разработки программного
обеспечения, требования к вычислительной системе. Была дана структура
разработанного приложения.

\pagebreak