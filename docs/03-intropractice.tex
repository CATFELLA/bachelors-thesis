\section*{ВВЕДЕНИЕ}
\addcontentsline{toc}{section}{ВВЕДЕНИЕ}

В 2019 году процессоры архитектуры ARM составляли 34\% от рынка процессоров, при этом занимая 90\% рынка мобильных процессоров \cite{arm_report}. С появлением процессоров M1 компании Apple большое число людей стали пользоваться компьютерами на основе архитектуры ARM в качестве персональных компьютеров. (чето написать типа им нужно использовать проприетарный софт который нельзя самому для себя пересобрать)

Программы собранные под архитектуру x86 не работают на таких компьютерах, необходим или статический транслятор, такой как Rosetta 2, или виртуальная машина поддерживающая необходимую архитектуру. Rosetta 2 не транслирует программы не предназначенные для macOS, для запуска программ созданных для Windows или Linux нужно использовать виртуальную машину. Еще одно ограничение статической трансляции --- наличие самомодифицирующегося кода и динамических библиотек, таким образом использование только статической трансляции не запустит любую программу. \cite{fast_bin}

Использование трансляции по своей природе несет дополнительные издержки при выполнении программы, как в памяти, так и в скорости выполнения. Цель этой работы --- исследование причин дополнительных издержек и попытка нахождения пути устранения этих издержек.

Для достижения поставленной цели необходимо решить следующие задачи:

\begin{itemize}[leftmargin=1.6\parindent]
%	\item [---] описать инструменты и технологии используемые для трансляции;
%	\item [---] проанализировать издержки трансляции;
	\item [---] реализовать оптимизирующий проход над промежуточным представлением;
	\item [---] встроить полученную реализацию в динамический транслятор FEX;
	\item [---] исследовать результаты работы оптимизирующего прохода;
	\item [---] исследовать корректность работы оптимизирующего прохода.
\end{itemize}
\pagebreak