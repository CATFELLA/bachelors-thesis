\section{Конструкторская часть}

страниц 10-15

В данном разделе описываются используемые структуры данных, проводится
подробное описание алгоритма (алгоритмов? а если я хочу написать про те что я не реализовал еще ??) "рассмотреть еще схему с джампом в рантайме") оптимизации динамической трансляции доступа к памяти. Проводится проектирование программной реализации алгоритма оптимизации динамической трансляции доступа к памяти.

Вообще говоря мой алгоритм расчитан на очень такую специфичную организацию кода, код разбит на строчки а эти строчки группируются в блоки. от этого и работаем типа.

"в рпз описать проблему с графами" агде

\subsection{Архитектура программного обеспечения}

ну вообще не обязательно ?

или обязательно...... 

\subsection{Используемые структуры данных}

$BlockInfo$ --- структура хранящая информацию об отдельном блоке транслированного кода. Включает в себя:

\begin{itemize}[leftmargin=1.6\parindent]
	\item[---] $State$ --- рассчитанное состояние блока, свидетельствует о состоянии регистра RBP, либо регистр не менялся, либо в нем адрес связанный со стеком, либо в нем адрес не связанный со стеком.
	\item[---] $StackNodes$ --- множество операций загружающих стековое значение;
	\item[---] $UnStackNodes$ --- множество операций загружающих не стековое значение;
	\item[---] $Predecessors$ --- множество блоков трансляции предшествующих этому;
	\item[---] $Visited$ --- флаг, показывает обработан ли блок.
\end{itemize}


\begin{comment}
{
	+  int State = NOT_CHANGED;
	+  std::set<OrderedNode*> StackNodes;
	+  std::set<OrderedNode*> UnStackNodes;
	+  std::vector<OrderedNode*> Predecessors;
	+  // std::vector<OrderedNode*> Successors;
	+  bool Visited = false;
	+};
+
\end{comment}

\subsection{Алгоритм оптимизации динамической трансляции доступа к памяти}

\textbf{Входные данные:} Множество блоков транслированного кода $IRBlocks$.

\textbf{Выходные данные:} Множество блоков транслированного кода $IRBlocks$ с оптимизированным доступом к памяти.

псевдокод

\subsection{Алгоритм control flow?}

\subsection{Алгоритм complex state?}

лень...

\subsection{Вывод}

Были описаны используемые структуры данных, дано описание алгоритмов используемых.. Был спроектирован алгоритм оптимизации динамической трансляции доступа к памяти.?

\pagebreak